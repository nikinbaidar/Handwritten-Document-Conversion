\documentclass[compress]{beamer} 

\usecolortheme{rose}

\usepackage{graphicx} 
\usepackage{float} 
\usepackage{xcolor} 
\usepackage{ulem}

\setbeamercolor{footline}{fg=black}
\setbeamerfont {footline}{series=\bfseries}

\definecolor{ForestGreen}{HTML}{00a550}

\beamertemplatenavigationsymbolsempty

\title{Handwritten Document Conversion - 1}
\author{Nikin Baidar \and Nimesh Gopal Pradhan}
\date{\today}

\begin{document}

\begingroup
\setbeamertemplate{footline}{}
\begin{frame}
  \titlepage
\end{frame}
\endgroup

\begingroup
\addtocounter{framenumber}{-1}
\setbeamertemplate{navigation symbols}{
  \usebeamerfont{footline}
  \usebeamercolor[fg]{footline}
  \insertframenumber
}

\iffalse

  \begin{frame}[t]
    \frametitle{Statement Of Problem}

    \vskip 10pt

    \begin{itemize}

      \item While digitization has preserved the conten        \vskip 10pt

      \item In spatial domain, we deal with images as it is. In frequency
        domain we deal with the \textcolor{ForestGreen}{rate at which the pixel values
        change} in the spatial domain. \vskip 10pt

      \item Image processing in \textcolor{red}{frequency domain} involves
        transforming an image in spatial domain into frequency domain using
        \textcolor{red}{Fourier transform}. \vskip 10pt

      \item Operations are performed in the frequency domain and the image is
        re-transformed into spatial domain for visualization.

    \end{itemize}

  \end{frame}

  \begin{frame}
    \frametitle{Image Processing Steps: Frequency Domain}

    \begin{figure}[H]
      \centering
      \caption{Block diagram of steps to be taken for image processing in the
      frequency domain.}
      \label{fig:}
    \end{figure}
  \end{frame}
  

  \begin{frame}
    \frametitle{Understanding Frequency in Images}

    \begin{figure}[H]
      \centering
      \caption{Illustration of image frequency}
      \label{fig:}
    \end{figure}

    \begin{itemize}
      \item Zebra pattern: High Frequency
      \item Sky : Low frequency
    \end{itemize}
    
  \end{frame}

  \begin{frame}
    \frametitle{Fourier Transform}

    \vskip -10pt

    \begin{figure}[H]
      \centering
    \end{figure}

    \vskip -10pt

    \begin{block}{2D Fourier transform:} 
    \begin{equation*}
      I(\omega,\psi) = \iint_{-\infty}^{+\infty} i(x,y) \,
      e^{-j2\pi(\omega x + \psi y)} dy\,dx
      \label{1D_fourier_transform}
    \end{equation*}
    \end{block}

  \end{frame}

  \begin{frame}
    \frametitle{2D Discrete Fourier Transform}

      Let the image data be $i(x, y)$, where $x$ represents the rows and has
      range $[0, 1, 2, \ldots M-1]$ and $y$ represents the columns and has
      range $[0, 1, 2, \ldots N-1]$. 

    \begin{block}{2D DFT:}
      
      \begin{equation}
        I(\omega, \psi) = \frac{1}{MN} \sum_{x=0}^{M-1} \sum_{y=0}^{N-1} i(x,y) \;
        e^{-j2\pi \big( \frac{\omega x}{M} + \frac{\psi y}{N} \big)}
      \end{equation}

    \end{block}

    \begin{block}
      {2D IDFT:}

    \begin{equation}
      i(x,y) = \sum_{x=0}^{M-1} \sum_{y=0}^{N-1} I(\omega, \psi) \;
      e^{j2\pi \big( \frac{\omega x}{M} + \frac{\psi y}{N} \big)}
    \end{equation}
    
    \end{block}

    \vskip 5pt

    Here, `$\omega$' corresponds to horizontal frequency and `$\psi$'
    corresponds to the vertical frequency.

  \end{frame}

  \begin{frame}
    \frametitle{Image Frequency Visualization}
    \begin{figure}[H]
      \centering
      \caption{Horizontal Frequency(L) \& Vertical Frequency(R)}
      \label{fig:}
    \end{figure}
  \end{frame}

  \begin{frame}
    \frametitle{2D DFT Examples}
    \begin{figure}[H]
      \centering
      \caption{Images in Spatial(L) and Frequency(R) Domains}
      \label{Images in Spatial(L) and Frequency(R) Domains}
    \end{figure}
    
  \end{frame}
  
  \begin{frame}
    \frametitle{2D DFT Examples}
    \begin{figure}[H]
      \centering
      \caption{Horizontal Frequency: Intensity image and Frequency image}
      \label{fig:-images-rect4160-png}
    \end{figure}
  \end{frame}

  \begin{frame}
    \frametitle{Adding Different Frequencies}
    \begin{figure}[H]
      \centering
      \caption{Sum of two different frequencies}
      \label{fig:-images-rectadded-png}
    \end{figure}
  \end{frame}
  
  \begin{frame}
    \frametitle{More 2D DFT Examples}
    \begin{figure}[H]
      \centering
      \caption{Vertical Frequency: Intensity image(L) and Frequency image(R)}
      \label{fig:-images-rect4160-png}
    \end{figure}
  \end{frame}

  \begin{frame}
    \frametitle{More 2D DFT Examples}
    \begin{figure}[H]
      \centering
      \caption{Frequencies in either direction}
    \end{figure}
  \end{frame}

  \begin{frame}

    As the images become more complex, it becomes difficult for us to visualize
    in our minds what it would look like in the frequency domain.
    
  \end{frame}

  \begin{frame}
    \frametitle{2D DFT: Rubik's Cube}

    \begin{figure}[H]
      \centering
      \caption{Intensity image (L) and Frequency Image(R)}
    \end{figure}
    
  \end{frame}

  \begin{frame}
    \frametitle{How we perceive things!}

    \begin{itemize}
      \item Humans vision system (rods and cons) does not process the pixel
        intensities to form an image.
        
      \item It performs the frequency analysis of the signals incident on the
        retina.

    \end{itemize}

    \begin{figure}[H]
      \centering
      \caption{High Frequency Image(L) \& Low frequency Image(R)}
      \label{fig:}
    \end{figure}
    
  \end{frame}

  \begin{frame}
    \frametitle{What do you see?}

    \begin{figure}[H]
      \centering
      \caption{Hybrid Image}
      \label{fig:-images-hybrid_image-png}
    \end{figure}

  \end{frame}

  \begin{frame}
    \frametitle{Why deal with the Frequency Domain?}

    Provided that the way we perceive things is dependent on frequencies,
    processing images in the frequency domain allows to: \vskip 10pt

    \begin{itemize}
      \item Further implement diverse image processing tasks \vskip 10pt
      \item Detect some features that cannot be detected in the spatial domain.
        i.e. to obtain new information so that a different analysis can be
        made. \vskip 10pt
      \item Certain processing techniques such as convolution is
        computationally more effective in the frequency domain.
    \end{itemize}

    \begin{block}{Frequency Domain Applications}
      The primary operations performed in the frequency domain is the
      application of high pass and low pass filters to images.
    \end{block}

  \end{frame}

  \begin{frame}
    \frametitle{Showcasing Frequency Domain Applications}

    \begin{figure}[H]
      \centering
      \caption{High Pass Filtering}
    \end{figure}
    
  \end{frame}

  \begin{frame}
    \frametitle{Showcasing Frequency Domain Applications}

    \begin{figure}[H]
      \centering
      \caption{Low Pass Filtering}
    \end{figure}
    
  \end{frame}
  \begin{frame}
    \frametitle{Code}

    \vskip -60pt
    
    \begin{itemize}
      \item \sout{Google Colab} Localhost
        \begin{itemize}
          \item Implementing 2D DFT in Python: Custom and Built-ins
          \item Conversion of image from the spatial domain to the frequency domain.
          \item Visualizing images in frequency domain in Python
        \end{itemize}
    \end{itemize}
  \end{frame}

  \setbeamertemplate{footline}{}

  \begin{frame}[noframenumbering]
    \vspace{10pt}
    \begin{center}
      \Huge \textit{The end}
    \end{center}
  \end{frame}
\endgroup

\fi

\begin{frame}
    \frametitle{Work planned in the previous week}
    \subsection{Work planned in the previous week}

    \begin{enumerate}
        \item Annotate more data
        \item Train detectron2 on newly annotated data for a more precise text
            segment detection.
        \item Explore measures to improve Tesseract's performance
    \end{enumerate}
    \vfill
\end{frame}

\begin{frame}
    \frametitle{Progress}
    \subsection{Progress}
    \begin{enumerate}
    \item \sout{Annotate more data}
            \begin{itemize}
                \item Downsampled examples to improve performance on quality
                    images.
                \item 393 out of 400 annotations are complete for the training set.
                \item 40 out of 100 annotations are complete for the test set.
            \end{itemize}
        \item Train detectron2 on newly annotated data for more precise text
            segment detection.
        \item \sout{Measures to improve Tesseract's performance}
            \begin{itemize}
                    \item Tesseract provides an \texttt{image\_to\_data} that
                        provides a \texttt{confidence\_score} for each text it
                        extracts from an image; for different image processing
                        techniques we select the one with the highest
                        confidence score.
                    \item We also pass the final output to some sort of spell
                        checker to improve overall performance
            \end{itemize}
        \item Exploring options for the web app
            \begin{itemize}
                \item Next + Py backend or Streamlit?
            \end{itemize}
     \end{enumerate}
    \vfill
\end{frame}

\begin{frame}
    \frametitle{Next Steps}
    \subsection{Next Steps}
    \begin{enumerate}
        \item Backlog from last week
        \item Explore deep learning models for contextual spell checker instead
            of employing plain spell checking.
        \item A more robust output; map the input image to a \texttt{.txt} file
        \item Backend for the web app and (at least) a CLI interface.
    \end{enumerate}

    
\end{frame}

\end{document}

The term F(0,0) is called the DC component of the image and is the average of all
pixel intensity values in the image and is the origin of the image in the
frequency domain.

Dot on the right is the actual frequency.

Dot on the left is the mirror of the dot on the right along the central dot.

